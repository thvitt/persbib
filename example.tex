\documentclass[ngerman,german]{persbib}
\usepackage{babel}
\usepackage[utf8]{inputenx}

\title{Beispiel für die \textsf{persbib}-Klasse}
\author{Thorsten Vitt}
\institution{Institut für Beispielkunde und Exemplifizierung}
\authoremail{tv+persbib@thorstenvitt.de}

\bibliography{example-bib.bib}

\begin{document}

\maketitle

\section{Einleitung}
\label{sec:einleitung}

Dies ist ein kleines Beispiel für einen mit der
\textsf{persbib}-Klasse für \emph{Perspektive Bibliothek}\footnote{%
  zu finden unter \url{http://perspektive-bibliothek.uni-hd.de}}
gesetzten Artikel.  Ein längeres Beispiel ist Mirjam Blümms
Beitrag\autocite{Bluemm2012} zur ersten Ausgabe der Zeitschrift.

\clearpage
\printbibliography

\end{document}


%%% Local Variables: 
%%% mode: latex
%%% TeX-master: t
%%% End: 
